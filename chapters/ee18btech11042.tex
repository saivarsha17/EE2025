\begin{enumerate}[label=\thesubsection.\arabic*.,ref=\thesubsection.\theenumi]

\numberwithin{equation}{enumi}
\numberwithin{figure}{enumi}
\item Constellation diagram of BPSK 
\begin{figure}[!h]
		\resizebox{\columnwidth}{!}{\begin{circuitikz}[american]


    \ctikzset{tripoles/mos style/arrows}
    \draw[<->] (-5,0)--(5,0)node[right]{$x$};
    
    \filldraw[black](-2.5,0) circle(2pt) node[below] {$s_1$};
    \filldraw[black](2.5,0) circle(2pt) node[below] {$s_0$};
    \filldraw[black](0,0) circle(2pt) node[below] {$o$};
    
    
    
    
    
    
    
    
    
    
    
    
    
    
    \end{circuitikz}}
\caption{Constellation diagram}
\label{fig:ee18btech11042_1}
\end{figure}
\item Encoding 
\newline 
We will encode bits as  symbols $s_0$ and $s_1$.Here, we will transmit $s_0$ if bit is 0 and we transmit $s_1$ if bit is 1.


\[ s =   \left\{
\begin{array}{cc}

    s_0 ,&     bit = 0 \\
    s_1, &     bit = 1 \\
\end{array}
\right .\]

 



\item Decision rule for BPSK.
\newline
Given symbols $s_0$ and $ s_1$ are equiprobable and assume symbol carries $\sqrt{E_b}$ per bit and consider a  additive white gaussian noise(AWGN) with mean 0 and variance $\frac{N_o}{2}$ and take  symbols as equiprobable. The received symbols can be:
\begin{align}
    y|s_0 = \sqrt(E_b) + n
    \label{eq:ee18btech11042_1}
\end{align}
\begin{align}
    y|s_1 = -\sqrt(E_b) + n
    \label{eq:ee18btech11042_2}
\end{align}
According to MAP detction rule, we will decode the received signal  as symbol s for which  p\brak{s|y} is more.
\begin{align}
    \hat{s} = max_{s \in  \cbrak{s_0,s_1}} p\brak{s|y}
    \label{eq:ee18btech11042_3}
\end{align}
\begin{align}
    \implies p\brak{s_0|y} \dec{s_0}{s_1} p\brak{s_1|y}
    \label{eq:ee18btech11042_4}
\end{align}
Using Bayes rule,
\begin{align}
    p\brak{s_0|y} = \frac{p\brak{y|s_0}p\brak{s_0}}{p\brak{y}}
    \label{eq:ee18btech11042_5}
\end{align}
\begin{align}
    p\brak{s_1|y} = \frac{p\brak{y|s_1}p\brak{s_1}}{p\brak{y}}
    \label{eq:ee18btech11042_6}
\end{align}
Since symbols are equi probable. p\brak{s_0} \&  p\brak{s_1} are equal.
\begin{align}
    \frac{p\brak{y|s_0}p\brak{s_0}}{p\brak{y}} \dec{s_0}{s_1}  \frac{p\brak{y|s_1}p\brak{s_1}}{p\brak{y}}
    \label{eq:ee18btech11042_7}
\end{align}
\begin{align}
    \implies p\brak{y|s_0} \dec{s_0}{s_1} p\brak{y|s_1}
    \label{eq:ee18btech11042_8}
\end{align}
\begin{align}
    \implies \frac{1}{\sqrt{2\pi}} \exp{-\frac{(y-\sqrt{E_b})^2}{\frac{N_o}{2}}}  \dec{s_0}{s_1}   
\\
    \frac{1}{\sqrt{2\pi}} \exp{-\frac{(y+\sqrt{E_b})^2}{\frac{N_o}{2}}}
    \label{eq:ee18btech11042_9}
\end{align}

\begin{align}
     \implies (y+\sqrt{E_b})^2 \dec{s_0}{s_1} (y - \sqrt{E_b})^2
     \label{eq:ee18btech11042_10}
\end{align}
\begin{align}
    \implies y \dec{s_0}{s_1} 0
    \label{eq:ee18btech11042_11}
\end{align}
The decision region of BPSK is:
\begin{align}
    y \dec{s_0}{s_1} 0
    \label{eq:ee18btech11042_12}
\end{align}
\begin{figure}[!h]
		\resizebox{\columnwidth}{!}{\begin{circuitikz}[american]


    \ctikzset{tripoles/mos style/arrows}
    \draw[<->] (-5,0)--(5,0)node[right]{$x$};
    \draw[-,dashed] (0,-5)--(0,5);
    \filldraw[black](-2.5,0) circle(2pt) node[below] {$s_1$};
    \filldraw[black](2.5,0) circle(2pt) node[below] {$s_0$};
    \filldraw[black](0,0) circle(2pt) node[below] {$o$};
    \coordinate[label = left:$D_1$]  (a) at (4,3);
    \coordinate[label = left:$D_2$]  (a) at (-4,3 );
    \end{circuitikz}
}
\caption{Decision region for BPSK}
\label{fig:ee18btech11042_2}
\end{figure}
\item Decoding 
\newline
Consider, y is the received symbol .Then, we need to decode this symbol into bits.Here,we will  use decision region  to decode into bits.
\[ bit =   \left\{
\begin{array}{cc}

    y>0 ,&     bit = 0 \\
    y<0, &     bit = 1 \\
\end{array}
\right .\]
So, if received symbol y $>$0 ,we decode that symbol into 0  and if y$<$0 we decode it as 1.
\item The following code has simulation of ber of BPSK.
\begin{lstlisting}
 codes/bpsk_ber.py
\end{lstlisting}
\item Fixed point code
\begin{description}[font=$\bullet$\scshape\bfseries]
\item Fixed point code is used to decrease execution time  and  to decrease no.of bytes used to store data.
\item Like,its easier to store a integer like 1000 than to store a floating number 13.356.
\item  In  computational  part of code, decimals take more time to  execute compared to  integers.

\end{description}

\begin{lstlisting}
 codes/bpsk_ber.m
\end{lstlisting}
In the above code, the computational part of code is,
\newline  a signal  is received so we need to find  no. of  -ve values  are in that signal.Since, stored values are in decimal.
We are converting into integers. 
\begin{description}[font=$\bullet$\scshape\bfseries]
\item So, first we are converting noise stored values into integer.
\end{description}

\begin{align}
    n1 = n*100000
\end{align}
\begin{description}[font=$\bullet$\scshape\bfseries]
\item then, converting   signal  energy $(E_b)$ into integer. 
\end{description}
\begin{align}
    sig = E_b*100000
\end{align}
\begin{description}[font=$\bullet$\scshape\bfseries]  
\item We are doing this because decimal addition are difficult than integer .

\end{description}





\end{enumerate}


